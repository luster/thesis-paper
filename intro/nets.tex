\subsubsection{Neural networks as computational graphs}

A neural network generally is a computational graph $G = (V, E)$ where the
vertices $V$ correspond to nodes of computation, and the edges $E$ correspond
to data flow paths connecting the computational nodes. We limit our discussion
to feed-forward or directed acyclic graphs, in other words graphs where given
a starting vertex $v$ we are unable to follow the directed edges away from it
and return to $v$. With this limitation in place we are able to focus on
stateless graphs, which treat input data examples independently of one another.

With the additional limitation that all computational nodes in the graph
perform differentiable operations, and that some nodes are parametric, we are
able to use automatic differentiation and a gradient descent like optimization
algorithm to optimize the parameters of the graph against a differentiable
loss function \cite{DBLP:journals/corr/BaydinPR15}. Automatic differentiation
is an alternative to numeric and symbolic differentiation, that is efficient,
accurate to machine precision, and scalable to arbitrary graphs consisting of
elementary operations. The core tenet of automatic differentiation is that
since graph vertices are differentiable we can perform the chain rule at each
node to build up gradients of the full graph. Several machine learning
frameworks, such as Theano and TensorFlow implement automatic differentiation,
allowing us to specify the functional form of the graph and getting gradients
at minimal cost \cite{bergstra-proc-scipy-2010,DBLP:journals/corr/AbadiABBCCCDDDG16}.

\subsubsection{Fully connected neural networks}
\label{sec:nn}

While we have introduced the idea that neural networks can be made up of
arbitrary computational nodes, there are several common types of nodes that
are used to build up a toolkit of functional forms at our disposal. First, let
us a consider a simple inner product with signature:
\begin{equation}
\langle \cdot, \cdot \rangle \colon \mathbb{R}^n \times \mathbb{R}^n \to \mathbb{R}
\end{equation}
And functional form:
\begin{equation}
\langle \mathbf{w}, \mathbf{x} \rangle = \sum\limits_{i = 1}^n w_ix_i
\end{equation}

We can interpret this inner product as a single node in our graph, with
parameters $\mathbf{w}$ and input vector $\mathbf{x}$. While this functional
form is nice, in that it's linear, and easy to define, it's limited by the
fact that it brings the rich space $\mathbb{R}^n \times \mathbb{R}^n$ down to
$\mathbb{R}$ which has limited representational power for problems of interest
in machine learning. Instead we'll follow a common design pattern, and create
an array of inner product nodes at a constant number of hops from the source node
in the graph, or as we'll refer to from here on, at a constant depth. In this case the signature would be:
\begin{equation}
g \colon \mathbb{R}^{n \times m} \times \mathbb{R}^n \to \mathbb{R}^m
\end{equation}
And functional form:
\begin{equation}
g\left(W, \mathbf{x}\right)_j = \sum\limits_{i=1}^n w_{ij}x_i
\end{equation}
which we recognize as ordinary matrix multiplication $W\mathbf{x}$, where $W
\in \mathbb{R}^{n \times m}$. We call this collection of nodes at a common
depth a layer. This type of matrix multiplication layer is generally referred to
as a linear, dense, or fully connected layer.

Layers can be connected together with the goal of creating a more powerful model. 
If we feed one dense layer into another directly we have have a functional form:
\begin{equation}
\mathbf{y} = W_n \cdots W_3W_2W_1\mathbf{x}
\end{equation}

where the $W_i$'s are all compatible. However this approach is actually
equivalent to a single matrix multiply $W\mathbf{x}$ where $W$ is the matrix
product of the $W_i$'s, so we did not actually achieve our goal of higher
representational power. Our model is still merely linear.

To address this issue the we introduce the addition of a non-linear
transformation $f(\cdot)$ (known as an activation function) at the output of
the matrix multiply. So instead our functional form would be:
\begin{equation}
\mathbf{y} = f(W_n \cdots f(W_3f(W_2f(W_1\mathbf{x})))\cdots)
\end{equation}
As long as $f(\cdot)$ is differentiable we are able to train our now highly
non-linear model with our methods of automatic differentiation and gradient
based optimization.

Finally we introduce a bias $\mathbf{b}$ so that each node in a layer is not
constrained to intercept zero. Therefore our full functional form for a single fully
connected layer with a non-linearity is:
\begin{equation}
\mathbf{y} = f\left(W\mathbf{x} + \mathbf{b}\right)
\label{dense}
\end{equation}
with $W \in \mathbb{R}^{n\times m}$ and $\mathbf{b} \in \mathbb{R}^m$.

A traditional activation function is the sigmoid function:
\begin{equation}
S(t) = \frac{1}{1 + e^{-t}}
\end{equation}

which maps $t \in \mathbb{R}$ (i.e the interval $[-\infty, \infty]$) to $S(t)$
on the interval $[0,1]$. This can allow the interpretation of $S(t)$ as a
probability. Using this activation function Cybenko proved the universal
approximation theorem \cite{cybenko1989approximation} which shows that neural
networks of the form in Figure \ref{densenn} can approximate any function with
appropriate domain and codomains given a large of enough number nodes in the
middle layer, generally referred to as a hidden layer.

\begin{figure}
\begin{center}
\begin{tikzpicture}[
plain/.style={
  draw=none,
  fill=none,
  },
net/.style={
  matrix of nodes,
  nodes={
    draw,
    circle,
    inner sep=5pt
    },
  nodes in empty cells,
  column sep=-0.5cm,
  row sep=-5pt
  },
>=latex
]
\matrix[net] (mat)
{
|[plain]| \parbox{1.3cm}{\centering Input\\layer} & |[plain]| \parbox{1.3cm}{\centering Hidden\\layer} & |[plain]| \parbox{1.3cm}{\centering Output\\layer} \\
& |[plain]| \\
|[plain]| & \\
& |[plain]| \\
|[plain]| & |[plain]| \\
& & \\
|[plain]| & |[plain]| \\
& |[plain]| \\
|[plain]| & \\
& |[plain]| \\
};
\foreach \ai [count=\mi ]in {2,4,...,10}
  \draw[<-] (mat-\ai-1) -- node[above] {$x_\mi$} +(-2cm,0);
\foreach \ai in {2,4,...,10}
{\foreach \aii in {3,6,9}
  \draw[->] (mat-\ai-1) -- (mat-\aii-2);
}
\foreach \ai in {3,6,9}
  \draw[->] (mat-\ai-2) -- (mat-6-3);
\draw[->] (mat-6-3) -- node[above] {Ouput} +(2cm,0);
\end{tikzpicture}
\end{center}
\caption[Feed-forward fully-connected network]{A single hidden layer feed-forward neural network. Each directed
connection has a scaling factor $w_{ij}$ associated with it; these entries make
up the $W$ matrix. Each circle, now referred to as a neuron, performs the
sum and activation functions corresponding to the matrix multiply, and
mapping by $f(\cdot)$ in Equation \ref{dense}. Layers in the middle of the
network are referred to as hidden layers because they are not directly observable
--- instead these are latent variables.}
\label{densenn}
\end{figure}

Figure \ref{densenn} shows a feed-forward network with one hidden-layer. The
outputs of the hidden layer are referred to as latent variables and are a new type of
representation for the input data $\mathbf{x}$. With each hidden-layer added to
the network increasingly complex data representations may be available, but the
networks will also tend to over-fit their training data as the number of
parameters increase. In order to combat over-fitting, and increase the
generalization performance of the network new types of architectures have been
proposed.


\subsubsection{Convolutional neural networks}
% The first convolutional neural network was proposed by Fukushima in 1980 \cite{conv}. 
% The idea of the network is that instead of creating a weight matrix $W$ with
% parameters that correspond to every entry in $\mathbf{x}$, that there would be a
% kernel of some small size that would be swept across across $\mathbf{x}$. This
% process is essentially a convolution. Mathematically this process is:
% \begin{equation}
% \mathbf{y} = f(\mathbf{x} \star \mathbf{w})
% \end{equation}
% Where $\star$ represents a discrete convolution, $\mathbf{w}$ is a
% one-dimensional kernel and $f(\cdot)$ is an activation function. This method can
% be extended to work for two and three-dimensional data representations, for
% example:
% \begin{equation}
% Y = f(X \star W)
% \end{equation}
% where $Y,X$ and $W$ are all matrices.
\subsubsection{Convolutions with holes}
\subsubsection{Tranposed convolutions}
\subsubsection{Additional techniques}
\paragraph{Dropout}
Various teams have demonstrated that in deep networks neurons can learn
complex co-adaptation schemes; that is deep neurons respond to ``mistakes'' in
shallower neurons. In order to prevent co-adaptation, so that each neuron
learns a meaningful representation, the dropout scheme has been proposed 
\cite{JMLR:v15:srivastava14a,DBLP:journals/corr/abs-1207-0580,dahl2013improving}.
Dropout is a masking process. In order to apply dropout to a layer during
training, a binary mask is applied across neurons. The mask is determined by
sampling a Bernoulli distribution with a parameter $p$ corresponding to the
probability that a neuron will be masked. When a neuron is masked its output
is considered fixed at zero. When the network is used for evaluation no masks
are applied and all neurons are connected.
\paragraph{Batch normalization}
\paragraph{Rectified linear units}
Various activations have been proposed on the basis of similarity to the
potential activations in actual biological neurons in human eyes. Recently the
rectified linear unit (ReLU) has demonstrated significant performance
increases for network generalization and increased training speed
\cite{glorot2011deep,dahl2013improving}. The ReLU activation is defined as
$\text{max}(0,x)$. In other words a ReLU activation forwards along any
positive inputs and sets and negative inputs to zero.
\paragraph{Exponential linear units}
\paragraph{Residual networks}
\paragraph{Minibatch training}
Traditionally there were two different approaches to optimizing neural
network parameters, the online approach and the batch approach. In the online
approach, the parameters of the network are updated after each exposure to a
training example and gradient calculation. In the batch approach, the network
is exposed to all of the training examples, the gradients are accumulated and
then the network's parameters are updated. This was long believed to be the
better approach, as the accumulated and averaged gradient was more likely to
be an estimate of the ``true'' gradient of the network towards an optimized
solution. It was later shown  that, while the batch mode may give a better
estimate of the gradient, due to the noise and its stochastic nature, the
online approach actually leads to a faster convergence time, in terms of
number of examples \cite{Wilson:2003:GIB:965268.965272}. This is because in
the stochastic approach the optimizer is less likely to get stuck in local
minima. An alternative approach that recently has become popular is the mini-
batch approach. In this approach some number of examples, say $N$, are exposed
to the network, the gradients are accumulated, and the parameters are updated.
In this case $N$ is much less than the total number of training examples
available. This approach, with serial computational resources really only
represents a decrease in training speed, because it is fairly similar to the
batch approach. The reason this approach has become popular lately is that
with the parallel resources afforded by modern high performance graphics
processing units (GPUs) the increase in example-wise training time becomes a
decrease in wall-time to train.

\paragraph{Intelligent parameter initialization}
Kaiming et al. have demonstrated an improved parameter initialization
technique specifically designed for neurons with ReLU activations
\cite{DBLP:journals/corr/HeZR015}. This approach derives a method that enables
extremely deep models comprised of ReLU neurons to converge rather than stall,
as they would with other initialization schemes. The initial parameters for
the convolutional layers are drawn from a zero mean normal distribution with a
standard deviation of  $\sqrt{2/n_l}$. Where $n_l = k^2c$. This corresponds to
the number of connections in the response for $k\times k$ kernels processing
$c$ input channels.

\paragraph{ADAM optimizer}
\paragraph{Style loss}
