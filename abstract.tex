% Traditional audio denoising systems are often linear time-invariant (LTI) and often require access to clean data to properly train to remove noise. Since clean audio is often unavailable, we build on a partitioned denoising autoencoder for denoising audio signals when clean examples are unavailable for training. In addition, the nonlinearity of a neural network architecture provides additional gains over standard linear models. We compare existing semi-supervised denoising systems as well as canonical supervised denoising autoencoders. We show that for moderate levels of noise, our autoencoder can outperform existing schemes.

In this thesis, we introduce a modified partitioned autoencoder for de-noising audio without access to clean data for training. Traditional linear time-invariant (LTI) systems such as the Wiener filter rely on power spectral density (PSD) estimates of desired signals and noise signals, which require some knowledge of the ground truth signals. One nonlinear approach in this area includes the use of de-noising autoencoders, which are one form of artificial neural networks (ANN). The nonlinearity of neural networks allow for more complex models to be made than LTI models. However, since de-noising autoencoders also require access to clean data and knowledge of the noise corruption process, we build on existing literature for a semi-supervised partitioned autoencoder that can perform de-noising without the clean signals during training. We compare existing semi-supervised de-noising systems as well as canonical supervised de-noising autoencoders. We show that for moderate levels of noise, our autoencoder outperforms existing schemes.
